\chapter{Dynamics}

\section{Car model derivation}
\label{a:dynamics}
The dynamics for a simple car takes on the characteristic form shown in equation \eqref{eq:car_characteristic}:
\begin{equation}
[M]\ddot{x}+[b]\dot{x}+[k]x=F
\end{equation}
However, the full (populated) dynamic models are significantly more complicated, and reproduced in full below.

\begin{center}
\begin{tabular}{| c | l |}
\hline
Notation & Description \\
\hline
mu & Unsprung mass; body mass \\
ms & Sprung mass; wheel mass \\
mf & Front tire mass \\
mr & Rear tire mass \\
\hline
Ix & X Inertia; roll \\
Iy & Y Inertia; pitch \\
\hline
ku & Unsprung spring; suspension spring \\
ks & Sprung spring; tire spring constant \\
\hline
bu/cu & Unsprung damper; suspension damping \\
bs/cs & Sprung damper; tire damping constant \\
\hline
\end{tabular}
\end{center}

\subsection{Quarter car}
\begin{equation} \label{eq:quarter_car_m}
	[M] = \left[
		\begin{array}{c c}
		ms & 0 \\
		0 & mu \\
		\end{array}
	\right]
\end{equation}

\begin{equation} \label{eq:quarter_car_b}
	[b] = \left[
		\begin{array}{c c}
		bs & -bs \\
		-bs & (bs+bu) \\
		\end{array}
	\right]
\end{equation}

\begin{equation} \label{eq:quarter_car_k}
	[k] = \left[
		\begin{array}{c c}
		ks & -ks \\
		-ks & (ks+ku) \\
		\end{array}
	\right]
\end{equation}

\subsection{Half car}
\begin{equation} \label{eq:half_car_m}
	[M] = \left[
		\begin{array}{c c c c}
		ms & 0 & 0 & 0 \\
		0 & Ix & 0 & 0 \\
		0 & 0 & mf & 0 \\
		0 & 0 & 0 & mf \\
		\end{array}
	\right]
\end{equation}





\subsection{Full car}



\section{Car model parameter values}
The simulation was tested with several different car parameters from normal sedans and buses to R.C. trucks. However, most of the work was done using the parameters for the sedan found in Vehicle Dynamics \cite{book:jazar}. The specific values used are reproduced below. The values in parenthesis describe a more full car without the geometric center and center of mass co-located.

\subsection{Mass values}
\begin{center}
\begin{tabular}{| l | c |}
\hline
Object & Mass (kg) \\
\hline
Body mass & 840 \\
Front wheel mass & 53 \\
Rear wheel mass & 53 (76) \\
Roll inertia & 820 \\
Pitch inertia & 1100 \\
\hline
\end{tabular}
\end{center}


\subsection{Vehicle lengths}
\begin{center}
\begin{tabular}{| l | c |}
\hline
Object & Length (m) \\
\hline
Center of mass to front wheel & 1.4 \\
Center of mass to rear wheel & 1.4 (1.47) \\
Center of mass to left wheel & 0.7 (0.75) \\
\hline
\end{tabular}
\end{center}

\subsection{Spring values}
\begin{center}
\begin{tabular}{| l | c |}
\hline
Object & Stiffness (N/m) \\
\hline
Front suspension & 10,000 \\
Rear suspension & 10,000 (13.000) \\
Front tire & 200,000 \\
Rear tire & 200,000 \\
\hline
\end{tabular}
\end{center}

\subsection{Damping values}
\begin{center}
\begin{tabular}{| l | c |}
\hline
Object & Damping (N*s/m) \\
\hline
Front suspension & 9,600 \\
Rear suspension & 9,600 \\
Tires & 0 \\
\hline
\end{tabular}
\end{center}


\subsection{Miscellaneous values}
\begin{center}
\begin{tabular}{| l | c |}
\hline
Object & Value \\
\hline
Maximum speed & 13 m/s \\
\hline
\end{tabular}
\end{center}

