\chapter{Methodology}
\section{Potential fields}
The following algorithm to control a convoy was developed and implemented on three Khepera III robots. The trajectory of motion was generated using a version of the potential fields algorithm described in the previous section.

The aim was to design a control law for n number of unconnected wheeled autonomous robots to follow a lead robot's trajectory generated from the potential field algorithm, while maintaining a constant distance from each other. The control algorithm utilizes the potential field force equation \eqref{eq:potential_fields} to create a matrix of possible configurations for the movement of the robot. The robots' trajectories are defined by the following control equations:
\begin{equation} \label{eq:pot_lead}
	q_n=q+-K*F(q)q
\end{equation}
\begin{equation} \label{eq:pot_follow}
	q_{n(r)}=q_{(r+1)}-o
\end{equation}

Equation \eqref{eq:pot_lead} determines the lead robots next position ($q_n$) based on its current position ($q$) and the potential field force equation \eqref{eq:potential_fields}. The position $q$ and $q_n$ are 2x1 matrices of the form $[x y]^T$. The following robots' new positions ($q_{n(r)}$) are determined by equation \eqref{eq:pot_follow} for a simple leader-follower simulation. The offset from the lead robot ($q_{(r+1)}$) is $o$, a 2x1 matrix of the form $[x y]^T$ representing the straight-line $x$ and $y$ distance between the two robots.

In addition to a simple leader-follower simulation, a second simulation with a smoothing algorithm was implemented in MATLAB. The smoothing control algorithm accomplishes two goals; first, if a following robots leader disappears (i.e. breaks down) the follower becomes the leader and reaches its goal. Second, the following robots are able to follow a smoother trajectory to the goal. A smoother trajectory has less radical turns and directional adjustments required by the robot. The more follower robots, the smoother the final trajectory becomes. This algorithm uses these two equations for the follower robots:
\begin{equation} \label{eq:pot_follow_force}
	F(q) =
	\begin{cases}
		r=1, & -\nabla(U_a(q)+U_r(q) \\
		r>1, & -\nabla(q_{(r+1)}+U_r(q)
	\end{cases}
\end{equation}
\begin{equation} \label{eq:pot_follow_q}
	q_{n(r)}=q_r+-K*F(q)\dot{q}
\end{equation}

Equation \eqref{eq:pot_follow_force} uses the same potential field algorithm as the other simulation except for follower robots, where instead of the attractive field being assigned by the goal position; the attraction is defined by the robot's respective leader's position. For each time step, the robot's leader's position becomes the attractive force \eqref{eq:pot_follow_force} and a new position $q_{n(r)}$ is calculated based on it \eqref{eq:pot_follow_q}. Using this method, the following robots follow smoother trajectories because they react to the information learned by their leader about the terrain. It also solves the issue of a leader robot breaking down; the following robot becomes the new leader and successfully reaches the goal.

The Khepera III's that were used to implement the potential fields simulation use the standard differential drive method. The kinematic equations for this type of robot are:
\begin{eqnarray} \label{eq:diff_drive}
	\dot{x_i} &=& v_i(t)\cos\theta_i\\
	\dot{y_i} &=& v_i(t)\sin\theta_i\\
	\dot{\theta_i} &=& \omega_i(t)
\end{eqnarray}

where $(x_i,y_i)$ are the coordinates of the reference point of i-th robot in the Cartesian frame of reference. $\theta_i$ is its orientation angle with respect to the positive x-axis. $v_i$ and $\omega_i(t)$ are the linear and angular velocities, respectively.

\section{Robot tracking}
To calculate the current location of the robot in the world based off the current ground elevation and/or disturbances, the dynamics of the system are required. The dynamics of car based systems take on the following characteristic form:
\begin{equation} \label{eq:car_characteristic}
[M]\ddot{x}+[b]\dot{x}+[k]x=F
\end{equation}

where $[M]$ are the mass coefficients, $[b]$ are the damping coefficients, and $[k]$ are the spring coefficients. The full realization of this characteristic equation may be seen in appendix \ref{a:dynamics}.

Knowing the position and changes in the center of mass of the vehicle, equation \eqref{eq:car_characteristic} may also be used to calculate the inverse dynamics of the system. This allows for a following vehicle to be able to calculate the ground input on the leading vehicle based off it's movements.

